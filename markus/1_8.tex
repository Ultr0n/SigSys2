\section*{Aufgabe 1.8}
\subsection*{Angabe}
Gegeben ist ein unvollständiges Spektrum eines \textit{reellen, periodischen, mittelwertfreien (ohne Gleichanteil) Signals} mit Periode $N=4$.
\begin{center}
\resizebox{200pt}{100pt}{%
	\begin{tikzpicture}
		\begin{axis}[
			xmin = 0, xmax = 7,
			ymin = -2, ymax = 3,
			axis x line=middle,
			axis y line=left,
			ylabel = {$c_k$},
			xlabel = $k$,
       		xtick={0,1,...,7},
			]
			\addplot+[ycomb,blue,thick,domain=0:1,samples=2,mark=o] table {1_8.dat};
		\end{axis}
	\end{tikzpicture}
}\end{center}
Bestimmen Sie aus diesen Angaben das Zeitsignal $x\leftn[n\right]$, sowie die fehlenden Fourierreihenkoeffizienten $c_0$ und $c_3$.
\subsection*{Lösung}
Unser Signal hat offensichtlich die gerade Periode $N=4$. Aus dem Buch (Seite 23-25) wissen wir:
\begin{hint}
Damit $x\leftn[n\right]$ reell ist, muss $c_k$ daher die folgenden Eigenschaften besitzen:\\
\textbf{Gerade Periodendauer} $N$:
\begin{align*}
	c_0, c_{N/2}&,\quad \text{reell}\\
	c_k=c^*_{N-k}&,\quad k=1,2,...,\frac{N}{2}-1
\end{align*}
Als vereinfachte \textbf{Fourierreihendarstellung für reellwertige Signale} $x\leftn[n\right]$ erhalten wir
\[
	x\leftn[n\right]=c_0+c_{N/2}(-1)^n+2 \Re e 
	\left\{\sum_{k=1}^{\frac{N}{2}-1}c_k e^{j\frac{2\pi}{N}kn}\right\}
\]
\end{hint}
Für unser Signal gilt:
\begin{align*}
	x\leftn[n\right] &= c_0+c_\frac{4}{2} (-1)^n + 2 \Re e
	\left\{\sum_{k=1}^{\frac{4}{2}-1}c_k e^{j\frac{2\pi}{N}kn}\right\} \\
	&= \underbrace{c_0}_{Gleichanteil} + c_2 (-1)^n + 2 \Re e \left\{ c_1 e^{j\frac{2\pi}{4}n} \right\}\\
	&= 2 (-1)^n   + 2 \Re e \left\{ e^{j\frac{\pi}{2}n} \right\}
\end{align*}
Da unser Signal periodisch ist, müssen wir nur die Werte für $n=0,1,...,3$ berechnen:
\begin{center}
	\begin{tabular}{| c | c | c | c | c |}
	\hline
	\textbf{n} & 0 & 1 & 2 & 3 \\ \hline
	\textbf{x[n]} & 4 & -2 & 0 & -2 \\
	\hline
	\end{tabular}
\end{center}
$c_0$ ist der Gleichanteil und somit 0. Für $c_3$ gilt bezüglich der Symmetrie $c_3=c_1=1$.