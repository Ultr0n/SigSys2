\section*{Aufgabe 1.4}

\subsection*{Angabe}
 
Welche Grundperiode hat das zeitdiskrete Signal
\[
	x\leftn[n\right]=e^{j\frac{2\pi}{N}nk}
\]
in Abhängigkeit von den ganzzahligen Größen $k$ und $N$?

\subsection*{Lösung}
Für ein periodisches Signal $x\leftn[n\right]$ mit der Periode $N$ gilt allgemein
\[
	x\leftn[n\right]=x\leftn[n+m\cdot N\right], \; m\in \mathbb{Z}
\]
Wenden wir nun diese Eigenschaft auf unser zeitdiskretes Signal an:
\begin{align*}
	x\leftn[n+N\right] &= e^{j\frac{2\pi}{N}(n+N)k}\\
	&= e^{j\frac{2\pi}{N}nk}e^{j\frac{2\pi}{N}Nk}\\
	&= e^{j\frac{2\pi}{N}nk}\underbrace{e^{j2\pi k}}_{=1}\\
	&= e^{j\frac{2\pi}{N}nk}
\end{align*}
In der Funktion bezeichnen wir $N$ als die \textbf{Periodendauer der Grundfrequenz}. Durch die Variation von $k=0,1,...,N-1$ erzeugen wir ganzzahlige Vielfache der Grundfrequenz, also harmonische Exponentialschwingungen bzw. Oberschwingungen.