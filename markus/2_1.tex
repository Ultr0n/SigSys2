\section*{Aufgabe 2.1}
\subsection*{Angabe}
Beweisen Sie für jedes der durch Eingangs/Ausgangsbeziehungen gegebenen Systeme die Gültigkeit oder Ungültigkeit der Systemeigenschaften \textit{Linearität, Zeitinvarianz, Kausalität, Stabilität.} Wo es möglich ist, geben Sie die Impulsantwort $h\leftn[n\right]$ des Systems an.
\begin{enumerate}[a)]
	\item $y[n] = x[n]-x[n-1]$
	\item $y[n] = x[n]x[n-1]$
	\item $\displaystyle{y[n] = \sum_{k=n-2}^{n+4}x[k]}$
	\item $y[n] = x[n] + x[-n]$
	\item $y[n] = x[2n]$
	\item $y[n] = \frac{1}{n+0.5}x[n]$
	\item $y[n] = x[n-1]+x[n]-x[n+1]$
\end{enumerate}
\subsection*{Theorie}
\begin{hint}
	Für lineare Systeme gilt das \textbf{Superpositionsprinzip}.\\
	\[
		y\leftn[n\right] = \mathcal{T} \left\{ ax_1 \leftn[n\right] + bx_2 \leftn[n\right] \right\}
		= a\mathcal{T} \left\{ x_1\leftn[n\right] \right\} + b\mathcal{T} \left\{ x_2 \leftn[n\right] \right\}
		\quad \forall n,a,b
	\]
\end{hint}
\begin{hint}
	Ein System heißt dann zeitinvariant, wenn für jede beliebige Zeitverschiebung um $n_0$ gilt:
	\[
		\mathcal{T}\{x[n-n_0]\} = y[n-n_0]
	\]
	Für die Zeitinvarianz muss das Ausgangssignal den Zeitbezug zum Eingangssignal beibehalten und identisch reagieren. Dieses Prinzip wird auch als Verschiebungsprinzip bezeichnet.
\end{hint}
\begin{hint}
	Bei \textbf{kausalen Systemen} eilt die Systemantwort der Systemanregung nicht voraus. Die Systemantwort für jeden Zeitindex $n_0$ hängt nur von Signalwerten $x\leftn[n\right]$ zu Zeitpunkten $n\le n_0$, also von vergangenen Eingangssignalwerten, ab. \\
	Daraus folgt mit der Faltungssume:
	\[
		y\leftn[n_0\right] = \sum_{k=-\infty}^{\infty} h\leftn[k\right]x[\underbrace{n_0-k}_{\le n_0}]
	\]
	als Konsequenz für die \textbf{Impulsantwort kausaler Systeme}
	\[
		h[n] = 0 \quad \text{für} \quad n < 0
	\]
\end{hint} \clearpage
\begin{hint}
	Ein System, für das folgende Gleichung erfüllt ist, wird \textbf{BIBO-stabiles System} (BIBO = Bounded Input Bounded Output) genannt.
	\[
		\sum_{k=-\infty}^{\infty}|h\leftn[n\right]|<\infty
	\]
	Filter mit einer Impulsantwort endlicher Dauer sind daher immer stabil.
\end{hint}
\subsection*{Lösung}
\begin{enumerate}[a)]
	\item $y[n] = x[n]-x[n-1]$ \\
	Zuerst testen wir die Linearität mit dem Superpositionsprinzip. Das Eingangs/Ausgangsverhältnis ist definiert mit:
	\[
		y[n] = \mathcal{T}\{x[n]\} = x[n]-x[n-1]
	\]
	Für die \textbf{Linearität} muss also gelten:
	\[
		y\leftn[n\right] = \mathcal{T} \left\{ ax_1 \leftn[n\right] + bx_2 \leftn[n\right] \right\}
		= a\mathcal{T} \left\{ x_1\leftn[n\right] \right\} + b\mathcal{T} \left\{ x_2 \leftn[n\right] \right\}
		\quad \forall n,a,b
	\]
	Wenden wir das auf unser Verhältnis an:
	\begin{align*}
		\mathcal{T}\{ ax[n] + bx[n] \} &= a\mathcal{T} \{ x[n] \} + b\mathcal{T} \{ x[n] \} \\
		ax[n] + bx[n] - ax[n-1] - bx[n-1] &= a(x[n]-x[n-1]) + b(x[n]-x[n-1]) \\
		ax[n] - ax[n-1] + bx[n] - bx[n-1] &= ax[n] - ax[n-1] + bx[n] - bx[n-1] \\
		&q.e.d.
	\end{align*}
	Für die \textbf{Zeitinvarianz} überprüfen wir das Verschiebungsprinzip:
	\begin{align*}
		\mathcal{T} \{ x[n-n_0] \} &= y[n-n_0] \\
		x[n-n_0] - x[n-n_0-1] &= y[n-n_0] \\
		&q.e.d.
	\end{align*}
	Bei \textbf{kausalen Systemen} hängt die Systemantwort nur von vergangenen Eingangssignalwerten ab. Diese Eigenschaften können wir direkt herauslesen. \\
	Die Impulsantwort $h[n]$ bekommen wir, indem wirals Eingangsignal $x[n] = \delta [n]$ den Dirac-Impuls verwenden.
	\[
		h[n] = \delta [n] - \delta [n-1]
	\]
	Die Impulsantwort hat eine endliche Dauer und ist \textbf{immer stabil}.

	\item $y[n] = x[n]x[n-1]$
	\subsubsection*{Linearität}
		\begin{align*}
			\mathcal{T}\{ ax[n] + bx[n] \} &= a\mathcal{T} \{ x[n] \} + b\mathcal{T} \{ x[n] \} \\
			(ax[n] + bx[n])(ax[n-1] + bx[n-1]) &\neq ax[n]x[n-1] + bx[n]x[n-1] \\
			x[n]x[n-1](a^2+2ab+b^2) &\neq x[n]x[n-1](a+b)
		\end{align*}
		Das System ist somit nicht linear.
	\subsubsection*{Zeitinvarianz}
		\begin{align*}
			\mathcal{T} \{ x[n-n_0] \} &= y[n-n_0] \\
			x[n-n_0]x[n-n_0-1] &= y[n-n_0] \\
			&q.e.d.
		\end{align*}
	 	Das System ist zeitinvariant.
 	\subsubsection*{Kausalität}
	 Die Kausalität ergibt sich aus der Definition. 
	\subsubsection*{Stabilität}
	 Für die Impulsantwort gilt:
	 \[
	 	h[n] = \delta [n] \delta[n-1]
	 \]
	 Sie hat endliche Dauer und ist stabil.
	\item $\displaystyle{y[n] = \sum_{k=n-2}^{n+4}x[k]}$
	\subsubsection*{Linearität}
		\begin{align*}
			\mathcal{T} \{ ax[n] + bx[n] \} &= a\mathcal{T} \{ x[n] \} + b \mathcal{T} \{ x[n] \} \\
			\sum_{k=n-2}^{n+4}(ax[k]+bx[k]) &= a\sum_{k=n-2}^{n+4}x[k]+b\sum_{k=n-2}^{n+4}x[k] \\
			a\sum_{k=n-2}^{n+4}x[k]+b\sum_{k=n-2}^{n+4}x[k] &= a\sum_{k=n-2}^{n+4}x[k]+b\sum_{k=n-2}^{n+4}x[k] \\
			&q.e.d.
		\end{align*}
		Das System ist linear.
	\subsubsection*{Zeitinvarianz}
		\begin{align*}
			\mathcal{T} \{ x[n-n_0] \} &= y[n-n_0] \\
			\sum_{k=n-n_0-2}^{n-n_0+4}x[k] &= y[n-n_0]
		\end{align*}
		Das System ist zeitinvariant.
	\subsubsection*{Kausalität}
		Die Kausalität ist hier nicht gegeben, da $y[n]$ unter anderem von $x[n-2]$ abhängt.
	\subsubsection*{Stabilität} 
		Die Impulsantwort ist:
		\[
			h[n] = \sum_{k=n-2}^{n+4}\delta [k] = \delta [n-2] + \delta [n-1] + \delta [n] + \delta [n+1] + \delta [n+2] + \delta [n+3] + \delta [n+4]
		\]
		Laut Beispielsammlung-Lösung wäre dies:
		\[
			h[n] = \sigma[n+4]\cdot \sigma[n-2]
		\]
		Meiner Meinung nach wäre
		\[
			h[n] = \sigma[n+4]-\sigma[n-2]
		\]
		genauso korrekt.
	\item $y[n] = x[n] + x[-n]$
	\subsubsection*{Linearität}
		\begin{align*}
			\mathcal{T} \{ ax[n]+bx[n] \} &= a\mathcal{T} \{ x[n] \}+ b \mathcal{T} \{ x[n] \} \\
			ax[n]+bx[n]+ax[-n]+bx[-n] &= a(x[n]+x[-n])+b(x[n]+x[-n]) \\
			a(x[n]+x[-n])+b(x[n]+x[-n]) &= a(x[n]+x[-n])+b(x[n]+x[-n]) \\
			&q.e.d.
		\end{align*}
	\subsubsection*{Zeitinvarianz}
		\[ y[n] = x[n] + x[-n] = \mathcal{T} \{ x[n] \} \]
		\[ \mathcal{T} \{ x[n-n_0] \} = x[n-n_0] + x[-n+n_0] \neq y[n-n_o] = x[n-n_0] + x[-n-n_0] \]
		Das System ist nicht zeitinvariant.
	\subsubsection*{Kausalität}
		Setzen wir für $n$ Werte $<0$ ein, würde unser Ausgangssignal $y[n]$ von Werten $n_0 > n$ abhängen. Das System ist nicht kausal.
	\subsubsection*{Stabilität}
		Wenn $|x[n]|<M$, dann gilt auch $|x[n]+x[-n]| < M$ und somit $|y[n]|<M$. Das System ist stabil.
	\item $y[n] = x[2n]$
	\subsubsection*{Linearität}
		\begin{align*}
			\mathcal{T} \{ ax[n]+bx[n] \} &= a\mathcal{T} \{ x[n] \}+ b \mathcal{T} \{ x[n] \} \\
			ax[2n]+bx[2n] &= ax[2n]+bx[2n] \\
			&q.e.d.
		\end{align*}
	\subsubsection*{Zeitinvarianz}
		\[ y[n] = x[2n] = \mathcal{T} \{ x[n] \} \]
		\[ \mathcal{T} \{ x[n-n_0] \} = x[2n-2n_0] \neq y[n-n_o] = x[2n-n_0] \]
		Das System ist nicht zeitinvariant.
	\subsubsection*{Kausalität}
		Das System ist offensichtlich nicht kausal.
	\subsubsection*{Stabilität}
		Wenn $|x[n]|<M$, dann gilt auch $|x[2n]| < M$ und somit $|y[n]|<M$. Das System ist stabil.
	\item $y[n] = \frac{1}{n+0.5}x[n]$
	\subsubsection*{Linearität}
		\begin{align*}
			\mathcal{T} \{ ax[n]+bx[n] \} &= a\mathcal{T} \{ x[n] \}+ b \mathcal{T} \{ x[n] \}
		\end{align*}
	\subsubsection*{Zeitinvarianz}
	\subsubsection*{Kausalität}
	\subsubsection*{Stabilität}
	\item $y[n]=x[n-1]+x[n]-x[n+1]$
	\subsubsection*{Linearität}
		\begin{align*}
			\mathcal{T} \{ ax[n]+bx[n] \} &= a\mathcal{T} \{ x[n] \}+ b \mathcal{T} \{ x[n] \}
		\end{align*}
	\subsubsection*{Zeitinvarianz}
	\subsubsection*{Kausalität}
	\subsubsection*{Stabilität}
\end{enumerate}