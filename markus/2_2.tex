\section*{Aufgabe 2.2}
\subsection*{Angabe}
	Ein lineares, zeitinvariantes System (LTI-System) habe eine Impulsantwort $h[n]=\alpha ^n \sigma [n]$. Berechnen Sie die Systemantworten $y[n]$ auf folgende Eingangssignale:
	\begin{enumerate}[a)]
		\item $x[n]=\sigma [n]$
		\item $x[n]=\sigma [-n]$
		\item $x[n]=\sigma [n]+\sigma [-n+N]-1$
		\item $x[n]=\sigma [n+N]\sigma [-n+N]$
		\item $x[n]=\beta ^n \sigma [n]$, auch für $\beta = \alpha, \quad |\alpha|<1, |\beta| < 1$
	\end{enumerate}
\subsection*{Theorie}
	\begin{hint}
		Bei zeitinvarianten Systemen tritt bei Verschiebung des Eingangssignals, lediglich eine Verschiebung des Ausgangssignals aus. Für lineare, zeitinvariante Systeme gilt deshalb die grundlegende Eingangs/Ausgangsrelation (Faltungssumme):
		\[
			y[n] = \sum_{k=-\infty}^{\infty}x[k]h[n-k]=\sum_{k=-\infty}^{\infty}x[n-k]h[k]
		\]
	\end{hint}
	\begin{hint}
		Die Sprungfunktion $\sigma [n]$ ist definiert mit:
		\[
			\sigma [n] = \left\{ 
							\begin{array}{ll} 
								1, & n \ge 0 \\ 
								0, & \mbox{sonst} 
							\end{array}  
						 \right.
		\]
	\end{hint}
	\begin{hint}
		Wichtige Summenformeln von Reihen
		\begin{align*}
			\sum_{n=0}^{N-1}a^n &= \left\{ \begin{array}{ll} N & a=1\\ \frac{1-a^N}{1-a} & sonst \end{array} \right.\\
			\sum_{n=0}^{\infty}a^n &= \frac{1}{1-a},  \quad |a|<1
		\end{align*}
	\end{hint}
\subsection*{Lösung}
	\begin{enumerate}[a)]
		\item $x[n]=\sigma [n]$ \\
			Für das Ausgangssignal $y[n]$ müssen wir das Eingangssignal $x[n]$ mit der Impulsantwort $h[n]$ falten. Wir können dabei zwischen folgenden zwei Varianten wählen:
			\[
				y[n] = \sum_{k=-\infty}^{\infty}x[k]h[n-k]=\sum_{k=-\infty}^{\infty}x[n-k]h[k]
			\]
			Da die Impulsantwort $\alpha ^n$ enthält, und ich $\alpha ^{n-k}$ vermeiden möchte, wähle ich die zweite Variante.
			\begin{align*}
				y[n]	&= \sum_{k=-\infty}^{\infty}x[n-k]h[k] \\
						&= \sum_{k=-\infty}^{\infty}\sigma [n-k] \alpha ^k \underbrace{\sigma [k]}_{0 \text{ für } k<0} \\
						&= \sum_{k=0}^{\infty}\underbrace{\sigma [n-k]}_{>0 \text{ für } n \ge k} \alpha ^k \\
						&= \sum_{k=0}^{n} \alpha ^k = \left\{ \begin{array}{ll} n\sigma [n] & \alpha=1\\ \frac{1-\alpha^{n+1}}{1-\alpha}\sigma [n] & sonst \end{array} \right.
			\end{align*}
			Das $\sigma [n]$ kommt daher, dass das Eingangssignal nur für positive n Werte definiert ist.
		\item $x[n]=\sigma [-n]$
		\item $x[n]=\sigma [n]+\sigma [-n+N]-1$
		\item $x[n]=\sigma [n+N]\sigma [-n+N]$
		\item $x[n]=\beta ^n \sigma [n]$, auch für $\beta = \alpha, \quad |\alpha|<1, |\beta| < 1$
	\end{enumerate}