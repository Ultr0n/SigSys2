\section*{Aufgabe 1.1}

\subsection*{Angabe}
 
Für das zeitbegrenzte Signal
	\begin{equation*}
		x\leftn[n\right] = \left\{\begin{array}{cl} 0, & n<0\\n, & 0\le n\le 10\\0, & n>10 \end{array} \right.
	\end{equation*}
zeichnen Sie folgende Signale:
	\begin{enumerate}[a)] 
		\item $y\leftn[n\right] = x\leftn[n+5\right]$
		\item $y\leftn[n\right] = x\leftn[-n+5\right]$
		\item $y\leftn[n\right] = x\leftn[2n\right]$
		\item gerades und ungerades Teilsignal von $x\leftn[n\right]$
		\item $y\leftn[n\right] = x\leftn[n+10\right]+x\leftn[-n+10\right]-10\delta\leftn[n\right]$
	\end{enumerate}

\subsection*{Lösung}
Die gegebene Funktion $x\leftn[n\right]$:\\
\begin{center}
\resizebox{200pt}{150pt}{%
	\begin{tikzpicture}
		\begin{axis}[
			domain=-10:20,
			axis x line=bottom,
			axis y line=middle,
			ylabel = {$x\leftn[n\right]$},
			xlabel = $n$,
       		xtick={-10,-5,...,20},
			]
			\addplot+[ycomb,blue,thick,domain=-10:-1,samples=10,mark=o] {0};
			\addplot+[ycomb,blue,thick,domain=0:10,samples=11,mark=o] {x};
			\addplot+[ycomb,blue,thick,domain=11:20,samples=10,mark=o] {0};
			
			\draw[blue,thick] ({axis cs:10,10}) -- ({axis cs:10,0});
		\end{axis}
	\end{tikzpicture}
}\end{center}\clearpage
\textbf{a)} $\mathbf{y\leftn[n\right] = x\leftn[n+5\right]}$\\
\begin{center}
\resizebox{200pt}{150pt}{%
	\begin{tikzpicture}
		\begin{axis}[
			domain=-10:20,
			axis x line=bottom,
			axis y line=middle,
			ylabel = {$y\leftn[n\right]=x\leftn[n+5\right]$},
			y label style={at={(-0.1,0.95)}},
			xlabel = $n$,
       		xtick={-10,-5,...,20},
			]
			\addplot+[ycomb,blue,thick,domain=-10:-6,samples=5,mark=o] {0};
			\addplot+[ycomb,blue,thick,domain=-5:5,samples=11,mark=o] {x+5};
			\addplot+[ycomb,blue,thick,domain=6:20,samples=15,mark=o] {0};
			
			\draw[blue,thick] ({axis cs:5,10}) -- ({axis cs:5,0});
		\end{axis}
	\end{tikzpicture}
}\end{center}
\textbf{b)} $\mathbf{y\leftn[n\right] = x\leftn[-n+5\right]}$\\
\begin{center}
\resizebox{200pt}{150pt}{%
	\begin{tikzpicture}
		\begin{axis}[
			domain=-10:20,
			axis x line=bottom,
			axis y line=middle,
			ylabel = {$y\leftn[n\right]=x\leftn[-n+5\right]$},
			xlabel = $n$,
       		xtick={-10,-5,...,20},
			]
			\addplot+[ycomb,blue,thick,domain=-10:-6,samples=5,mark=o] {0};
			\addplot+[ycomb,blue,thick,domain=-5:5,samples=11,mark=o] {-x+5};
			\addplot+[ycomb,blue,thick,domain=6:20,samples=15,mark=o] {0};
			
			\draw[blue,thick] ({axis cs:-5,10}) -- ({axis cs:-5,0});
		\end{axis}
	\end{tikzpicture}
}\end{center}
\textbf{c)} $\mathbf{y\leftn[n\right] = x\leftn[2n\right]}$\\
\begin{center}
\resizebox{200pt}{150pt}{%
	\begin{tikzpicture}
		\begin{axis}[
			domain=-10:20,
			axis x line=bottom,
			axis y line=middle,
			ylabel = {$y\leftn[n\right]=x\leftn[2n\right]$},
			y label style={at={(-0.1,1)}},
			xlabel = $n$,
       		xtick={-10,-5,...,20},
			]
			\addplot+[ycomb,blue,thick,domain=-10:-1,samples=10,mark=o] {0};
			\addplot+[ycomb,blue,thick,domain=0:5,samples=6,mark=o] {2*x};
			\addplot+[ycomb,blue,thick,domain=6:20,samples=15,mark=o] {0};
			
			\draw[blue,thick] ({axis cs:5,10}) -- ({axis cs:5,0});
		\end{axis}
	\end{tikzpicture}
}
\end{center}
\pagebreak

\textbf{d) gerades und ungerades Teilsignal von} $\mathbf{x\leftn[n\right]}$\\
Für die geraden und ungeraden Teile definieren wir zuerst allgemein:
\[
	x\leftn[n\right]=x_g\leftn[n\right]+x_{ug}\leftn[n\right]
\]
Nun erhalten wir durch umformen:
\begin{align*}
\text{Gerades Signal: } x_g\leftn[n\right]&=\frac{1}{2}\left(x\leftn[n\right]+x\leftn[-n\right]\right), \forall n\\
\text{Ungerades Signal: } x_u\leftn[n\right]&=\frac{1}{2}\left(x\leftn[n\right]-x\leftn[-n\right]\right), \forall n
\end{align*}
Daraus lassen sich nun die geraden und ungeraden Teilsignale berechnen (einfaches Einsetzen):\
\begin{figure}[!h]	%1.2) d - Gerades Teilsignal
\centering
	\begin{tikzpicture}[scale=0.8]
		\begin{axis}[
			domain=-10:20,
			axis x line=bottom,
			axis y line=middle,
			ylabel = {$x_g\leftn[n\right]=\frac{1}{2}\left(x\leftn[n\right]+x\leftn[-n\right]\right)$},
			y label style={at={(0.2,1.2)}},
			xlabel = $n$,
       		xtick={-10,-5,...,20},
			]
			\addplot+[ycomb,blue,thick,domain=-10:0,samples=11,mark=o] {-x/2};
			\addplot+[ycomb,blue,thick,domain=1:10,samples=10,mark=o] {x/2};
		\end{axis}
	\end{tikzpicture} \caption*{Gerades Teilsignal}
\end{figure}
\begin{figure}[!h]	%1.2) d - Ungerades Teilsignal
\centering
	\begin{tikzpicture}[scale=0.8]
		\begin{axis}[
			domain=-10:20,
			axis x line=middle,
			axis y line=middle,
			ylabel = {$x_u\leftn[n\right]=\frac{1}{2}\left(x\leftn[n\right]-x\leftn[-n\right]\right)$},
			y label style={at={(0.2,1.2)}},
			xlabel = $n$,
       		xtick={-10,-5,...,20},
			]
			\addplot+[ycomb,blue,thick,domain=-10:10,samples=21,mark=o] {x/2};
		\end{axis}
	\end{tikzpicture} \caption*{Ungerades Teilsignal}
\end{figure}\clearpage
\textbf{e)} $\mathbf{y\leftn[n\right] = x\leftn[n+10\right]+x\leftn[-n+10\right]-10\delta \leftn[n\right]}$\\
\begin{figure}[!h]	%1.2) e - Dreieck
\centering
	\begin{tikzpicture}[scale=0.8]
		\begin{axis}[
			domain=-10:20,
			axis x line=middle,
			axis y line=middle,
			ylabel = {$x_u\leftn[n\right]=x\leftn[n+10\right]+x\leftn[-n+10\right]-10\delta \leftn[n\right]$},
			y label style={at={(0,1.2)}},
			xlabel = $n$,
       		xtick={-10,-5,...,20},
			]
			\addplot+[ycomb,blue,thick,domain=-10:0,samples=11,mark=o] {x+10};
			\addplot+[ycomb,blue,thick,domain=0:10,samples=11,mark=o] {10-x};
		\end{axis}
	\end{tikzpicture}
\end{figure}