\begin{uebsp}
    \begin{Exercise}
        Jedes Signal $x[n]$ kann in ein gerades Signal $x_g[n]$ und in ein ungerades Signal $x_u[n]$ zerlegt werden. Zeigen Sie allgemein, dass 
        \begin{enumerate}[a)]
            \item das Produkt $x_g[n]x_u[n]$ ein ungerades Signal ist.
            \item für das ungerade Signal gilt:
                \[\sum_{n=-\infty}^\infty x_u[n]=0\]
            \item die Signalenergie gegeben ist durch
                \[\sum_{n=-\infty}^\infty x^2[n]=\sum_{n=-\infty}^\infty x_g^2[n]+\sum_{n=-\infty}^\infty x_u^2[n].\]
        \end{enumerate}
    \end{Exercise}
    \begin{Answer}
        \begin{enumerate}[a)]
            \item z.z.: das Produkt $x_g[n]\cdot x_u[n]$ ist ein ungerades Signal:\label{itm:ugmul}
                \begin{definition}\label{def:geradeungeradedef}
                        Ein Signal heißt \textbf{gerade}, wenn gilt:
                        \[x[-n]=x[n]\]
                        und ein Signal heißt \textbf{ungerade}, wenn gilt:
                        \[x[-n]=-x[n]\]
                    \end{definition}

                    \begin{uebsp_theory}
                        In \fref{def:geradeungerade} wurde der gerade und der ungerade Anteil eines Signals definiert:
                        \[x_g[n]=\frac{1}{2}(x[n]+x[-n])\]
                        \[x_u[n]=\frac{1}{2}(x[n]-x[-n])\]
                    \end{uebsp_theory}

                    somit müssen wir folgendes berechnen:
                \begin{eqnarray*}
                    x_g[n]\cdot x_u[n]&=&\frac{1}{2}(x[n]+x[-n])\cdot\frac{1}{2}(x[n]-x[-n])\\
                    &=&\frac{1}{4}(x[n]+x[-n])\cdot(x[n]-x[-n])\\
                    x_g[n]\cdot x_u[n]&=&\frac{1}{4}(x[n]^2-x[-n]^2)\\
                                      &&\text{Setzen wir nun $n=-n$:}\\
                    x_g[-n]\cdot x_u[-n]&=&\underbrace{\frac{1}{4}(x[-n]^2-x[n]^2)}_{x[-n]}=\underbrace{-\left(\frac{1}{4}(x[n]^2-x[-n]^2)\right)}_{-x[n]}\\
                \end{eqnarray*}
                Somit erhalten wir: $x[-n]=-x[n]$ und gerade das ist die Definition der ungeraden Funktion (siehe \fref{def:geradeungeradedef}). q.e.d.
        \item z.z.: $\sum_{n=-\infty}^\infty x_u[n]=0$\label{itm:sumu}
            Als erstes spalten wir die Summe auf:
            \[\sum_{n=-\infty}^\infty x_u[n]=\sum_{n=-\infty}^{-1} x_u[n]+\sum_{n=1}^\infty x_u[n]+x_u[0]\]
            Laut \fref{def:geradeungeradedef} gilt: $x[-n]=-x[n]$ formt man jedoch um, erhält man: $-x[-n]=x[n]$.

            Ersetzt man somit in der linken Summenformel $n$ durch $-n$, muss zusätzlich die linke Summenformel noch mit $-1$ multipliziert werden, damit wieder Gleichheit herrscht.
            \[\sum_{n=-\infty}^\infty x_u[n]=-\sum_{n=-\infty}^{-1} x_u[-n]+\sum_{n=1}^\infty x_u[n]+x_u[0]\]
            Dreht man nun die Vorzeichen der linken Summe um, bekommt man das folgende:
            \[\sum_{n=-\infty}^\infty x_u[n]=\cancel{-\sum_{n=1}^{\infty} x_u[n]}+\cancel{\sum_{n=1}^\infty x_u[n]}+x_u[0]\]
            Somit heben sich beide Summen auf und übrig bleibt:
            \[\sum_{n=-\infty}^\infty x_u[n]=x_u[0]\]
            Da es sich um eine ungerade Funktion handelt, muss jedoch gelten: $x_u[0]=0$ sonst wäre die Funktion nicht ungerade. (siehe Wikipedia Gerade und ungerade Funktionen)
            \[\sum_{n=-\infty}^\infty x_u[n]=x_u[0]=0\]
            q.e.d.
            \item z.z. $\sum_{n=-\infty}^\infty x^2[n]=\sum_{n=-\infty}^\infty x_g^2[n]+\sum_{n=-\infty}^\infty x_u^2[n].$\\
                Zuerst schreiben wir die Linke Seite um:
                \begin{uebsp_theory}
                    Jedes Signal besteht aus einem geraden und ungeraden Anteil (siehe \fref{def:geradeungerade}:
                    \[x[n]=x_g[n]+x_u[n]\]
                \end{uebsp_theory}
                \[\sum_{n=-\infty}^\infty x^2[n]=\sum_{n=-\infty}^\infty(x_g[n]+x_u[n])^2=\sum_{n=-\infty}^\infty(x_g^2[n]+2x_u[n]x_g[n]+x_u^2[n])\]
                \[\sum_{n=-\infty}^\infty x_u^2[n]=\sum_{n=-\infty}^\infty x_g^2[n]+\underbrace{\sum_{n=-\infty}^\infty 2x_u[n]x_g[n]}_{\sum_{n=-\infty}^\infty y_u[n]}+\sum_{n=-\infty}^\infty x_u^2[n]\]
                Wie unter Punkt \ref{itm:ugmul} bereits bewiesen, ist die Multiplikation von einem geraden und einem ungeraden Signal wieder ein ungerades Signal ($2x_u[n]x_g[n]=y_u[n]\;\Rightarrow\;\sum_{n=-\infty}^\infty 2x_u[n]x_g[n]=\sum_{n=-\infty}^\infty y_u[n]$). Somit erhalten wir:
                \[\sum_{n=-\infty}^\infty x_u^2[n]=\sum_{n=-\infty}^\infty x_g^2[n]+\underbrace{\sum_{n=-\infty}^\infty y_u[n]}_{=0}+\sum_{n=-\infty}^\infty x_u^2[n]\]
            In Punkt \ref{itm:sumu} haben wir bereits bewiesen, dass $\sum_{n=-\infty}^\infty x_u[n]=0$, somit gilt:
                \[\sum_{n=-\infty}^\infty x_u^2[n]=\sum_{n=-\infty}^\infty x_g^2[n]+\sum_{n=-\infty}^\infty x_u^2[n]\]
                q.e.d.
        \end{enumerate}
    \end{Answer}
\end{uebsp}
