\begin{uebsp}
\begin{Exercise}
Berechnen Sie die Gesamtimpulsantwort $h[n]$ des abgebildeten Systems, das aus einzelnen
LTI–Systemen zusammengesetzt ist.
\begin{center}
\begin{tikzpicture}[auto, node distance=2cm,>=latex']
    \node [input, name=input] {};
    \node [block, right of=input] (h1){$h_1[n]$};
    \node [point, right=0.5cm of h1] (split){};
    \node [block, below right=0.25cm and 0.5cm of split] (h3) {$h_3[n]$};
    \node [right=0.5cm of h3] (h35){};
    \node [block, right=0.5cm of h35] (h4) {$h_4[n]$};
    \node [block, above=1cm of h35] (h2) {$h_2[n]$};
    \node [sum, right=5cm of split] (sum) {+};
    \node [output, right=3cm of sum] (output) {};

    \draw [draw,->] (input) -- node {$x[n]$} (h1);
    \draw [draw] (h1) -- (split);
    \draw [draw,->] (split) |- (h2.west);
    \draw [draw,->] (split) |- (h3);
    \draw [draw,->] (h3) -- (h4);
    \draw [draw,->] (h4) -| node[above left] {-} (sum);
    \draw [draw,->] (h2) -| (sum);
    \draw [draw,->] (sum) -- node {$y[n]=(x*h)[n]$} (output);
\end{tikzpicture}
\end{center}
Dabei sei:
\begin{eqnarray*}
    h_1[n]&=&\left(\frac{1}{2}\right)^n\sigma[n]\\
    h_2[n]&=&(n+1)\sigma[n]\\
    h_3[n]&=&h_2[n]\\
    h_4[n]&=&\delta[n-1]
\end{eqnarray*}
\end{Exercise}
\begin{Answer}
    \begin{definition}[$\delta$-Einsimpuls]
        \[\delta[n]=\begin{cases}1 & n=0\\0 & n\neq 0\end{cases}\]
    \end{definition}
    \begin{definition}[$\sigma$-Sprungfunktion]
        \[\sigma[n]=\begin{cases}0 & \text{für }n<0\\1& \text{für }n\geq 0\end{cases}\]
    \end{definition}
    \begin{uebsp_theory}
        Das Ausgangssignal $y[n]$ eines linearen, zeitinvarianten Systems mit der Sprungantwort $h[n]$ und dem Eingangssignal $x[n]$ kann mit folgender Formel berechnet werden:
        \[y[n]=\sum_{k=-\infty}^\infty x[k]h[n-k]=\sum_{k=-\infty}^\infty x[n-k]h[k]\]
    \end{uebsp_theory}
    \begin{uebsp_theory}
        Bei der Kettenschaltung (Kaskadenschaltung (Reihenschaltung)) 2-er 
        stabiler LTI-Systeme mit Impulsantworten $h_1[n]$ und $h_2[n]$ folgt:
        \[h[n]=\sum_{k=-\infty}^\infty h_1[k]h_2[n-k]=\sum_{k=-\infty}^\infty h_1[n-k]h_2[k]\]
        $\Rightarrow$ Es ist die Reihenfolge der Systeme egal:
        \begin{center}
        \begin{tikzpicture}
            \matrix[row sep=0.5cm, column sep=0.3cm] {
                \node (inputa) {};&
                \node (h1a) [block] {$h_1[n]$};&
                \node (h2a) [block] {$h_2[n]$};&
                \node (outputa) {};&
                \node (equa) {=};&
                \node (inputb) {};&
                \node (h1b) [block] {$h_1[n]$};&
                \node (h2b) [block] {$h_2[n]$};&
                \node (outputb) {};&
                \node (equa) {=};&
                \node (inputc) {};&
                \node (h) [block] {$h[n]$};&
                \node (outputc) {};\\
            };
            \draw [draw,->] (inputa) -- (h1a);
            \draw [draw,->] (h1a) -- (h2a);
            \draw [draw,->] (h2a) -- (outputa);
            \draw [draw,->] (inputb) -- (h1b);
            \draw [draw,->] (h1b) -- (h2b);
            \draw [draw,->] (h2b) -- (outputb);
            \draw [draw,->] (inputc) -- (h);
            \draw [draw,->] (h) -- (outputc);

        \end{tikzpicture}
        \end{center}
    \end{uebsp_theory}
    \begin{uebsp_theory}
        Bei der Paralellschaltung 2-er LTI-Systeme mit Impulsantworten 
        $h_1[n]$ und $h_2[n]$ folgt:
        \[h[n]=h_1[n]+h_2[n]\]
        \begin{center}
        \begin{tikzpicture}
            \node [input, name=inputa] {};
            \node [point, right=0.5cm of inputa] (split){};
            \node [block, below right=0.25cm and 0.5cm of split] (h1) {$h_1[n]$};
            \node [block, above=0.5cm of h1] (h2) {$h_2[n]$};
            \node [sum, right=2.2cm of split] (sum) {+};
            \node [output, right of=sum] (outputa) {};
            \node [right=1cm of outputa] (equ){=};
            \node [input, right=1cm of equ] (inputb){};
            \node [block, right=0.5cm of inputb] (h){h[n]};
            \node [output, right of=h] (outputb) {};

            \draw [draw,->] (inputa) -- (split);
            \draw [draw,->] (split) |- (h1);
            \draw [draw,->] (split) |- (h2);
            \draw [draw,->] (h2) -| (sum);
            \draw [draw,->] (h1) -| (sum);
            \draw [draw,->] (sum) -- (outputa);
            \draw [draw,->] (inputb) -- (h);
            \draw [draw,->] (h) -- (outputb);

        \end{tikzpicture}
        \end{center}
    \end{uebsp_theory}

    \textbf{TODO: wieso ist $h_3$ und $h_4$ stabil?}
    
    Am besten man zerlegt das gesamte System in kleine Teile:
    \begin{enumerate}[i)]
        \item Zusammenfassen und Berechnen der Serienschaltung von $h_3[n]$ und 
            $h_4[n]$ als $h_{34}[n]$:
            \begin{center}
            \begin{tikzpicture}[auto, node distance=2cm,>=latex']
                \node [input, name=input] {};
                \node [block, right=0.5cm of input] (h3) {$h_3[n]$};
                \node [block, right=0.5cm of h3] (h4) {$h_4[n]$};
                \node [output, right=0.5cm of h4] (output) {};

                \draw [draw,->] (input) -- (h3);
                \draw [draw,->] (h3) -- (h4);
                \draw [draw,->] (h4) -- (output);
            \end{tikzpicture}
            \end{center}
            Aus der Formel für die Reihenschaltung folgt:
        \begin{eqnarray*}
            h_{34}[n]&=&\sum_{k=-\infty}^\infty h_3[k]h_4[n-k]=
                \sum_{k=-\infty}^\infty (k+1)\sigma[k]\cdot
                \underbrace{\delta[n-k-1]}_{=1\text{ für }k=n-1}\\
                h_{34}[n]&=&(n\cancel-1\cancel+1)\sigma[n-1]=n\sigma[n-1]
        \end{eqnarray*}
        \item Zusammenfassen und Berechnen der Paralellschaltung von $h_2[n]$ und 
            $h_{34}[n]$ als $h_{2||34}[n]$:
            \begin{center}
            \begin{tikzpicture}
                \node [input, name=inputa] {};
                \node [point, right=0.5cm of inputa] (split){};
                \node [block, below right=0.25cm and 0.5cm of split] (h34)
                {$h_{34}[n]$};
                \node [block, above=0.5cm of h34] (h2) {$h_2[n]$};
                \node [sum, right=2.2cm of split] (sum) {+};
                \node [output, right of=sum] (outputa) {};

                \draw [draw,->] (inputa) -- (split);
                \draw [draw,->] (split) |- (h1);
                \draw [draw,->] (split) |- (h2);
                \draw [draw,->] (h2) -| (sum);
                \draw [draw,->] (h1) -|node[above left] {-} (sum);
                \draw [draw,->] (sum) -- (outputa);
            \end{tikzpicture}
            \end{center}
            Aus der Formel für die Reihenschaltung folgt:
        \begin{eqnarray*}
            h_{2||34}[n]&=&h_2[n]-h_{34}[n]=(n+1)\sigma[n]-n\sigma[n-1]=\underbrace{n\sigma[n]-n\sigma[n-1]}_{n\delta[n]}+\sigma[n]\\
            h_{2||34}[n]&=&n\delta[n]+\sigma[n]
        \end{eqnarray*}
        \item Zusammenfassen und Berechnen der Serienschaltung von $h_1[n]$ und 
            $h_{2||34}[n]$ als $h[n]$:
            \begin{center}
            \begin{tikzpicture}[auto, node distance=2cm,>=latex']
                \node [input, name=input] {};
                \node [block, right=0.5cm of input] (h1) {$h_{1}[n]$};
                \node [block, right=0.5cm of h1] (h234) {$h_{2||34}[n]$};
                \node [output, right=0.5cm of h234] (output) {};

                \draw [draw,->] (input) -- (h1);
                \draw [draw,->] (h1) -- (h234);
                \draw [draw,->] (h234) -- (output);
            \end{tikzpicture}
            \end{center}
            \begin{definition}[Geometrische Reihe]
                \[\sum_{n=0}^\infty a^n=\frac{1}{1-a}\;\;\forall
                    |a|<1\;\;\;\;\text{ und }\;\;\;\;
                    \sum_{n=0}^{N-1} a^n=\begin{cases}N&a=1\\
                        \frac{1-a^N}{1-a}&\text{sonst }\forall N\geq0\end{cases}\]
            \end{definition}
            \begin{eqnarray*}
                h[n]&=&\sum_{k=-\infty}^\infty h_1[k]h_{2||34}[n-k]=
                \sum_{k=-\infty}^\infty \left(\frac{1}{2}\right)^k\sigma[k]
                ((n-k)\delta[n-k]+\sigma[n-k])\\
                h[n]&=&\sum_{k=-\infty}^\infty \left(\frac{1}{2}\right)^k\sigma[k]
                (n-k)\underbrace{\delta[n-k]}_{=1\text{ für
                }k=n}+\sum_{k=-\infty}^\infty
                \left(\frac{1}{2}\right)^k\sigma[k]\sigma[n-k]\\
                h[n]&=&\left(\frac{1}{2}\right)^n\sigma[n]\underbrace{(n-n)}_{=0}+\sum_{k=-\infty}^\infty
                \left(\frac{1}{2}\right)^k\underbrace{\sigma[k]}_{=1\text{ für
                }k\geq0\;\;}\underbrace{\sigma[n-k]}_{\;\;=1\text{ für }n\geq
            k}\fbox{\begin{minipage}{0.2\linewidth}Somit geht $\sum$ von
            $k=0$ bis $n$\end{minipage}}\\
            h[n]&=&\underbrace{\sum_{k=0}^n\left(\frac{1}{2}\right)^k}_{\text{Geom.
            Reihe}}=\frac{1-\left(\frac{1}{2}\right)^{n+1}}{1-\frac{1}{2}}=
            \frac{1-\left(\frac{1}{2}\right)^{n+1}}{\frac{1}{2}}=
            2-\cancel2\left(\frac{1}{2}\right)^{n\cancel{+1}}\\
            h[n]&=&2-\left(\frac{1}{2}\right)^n\;\;\;\forall
            n\geq0\;\;\;\Rightarrow\;\;\;h[n]=\left(2-\left(\frac{1}{2}\right)^n\right)\sigma[n]
            \end{eqnarray*}
        \end{enumerate}
    \end{Answer}
\end{uebsp}
