\begin{uebsp}
\begin{Exercise}
Sie sollen ein zeitdiskretes System untersuchen, von dem die Beziehung zwischen
Eingangssignal $x[n]$ und Ausgangssignal $y[n]$ gegeben ist:
\[y[n]=\sum_{k=n-1}^{n+1}(x[k+1]-x[k]+x[k-1])\]
\begin{enumerate}[a)]
    \item Prüfen Sie (mit Beweis), ob das System die folgenden Eigenschaften besitzt:
        \begin{enumerate}[1)]
            \item Linearität,
            \item Kausalität,
            \item Stabilität,
            \item Zeitinvarianz.
        \end{enumerate}
    \item Berechnen und skizzieren Sie die Impulsantwort und die Übertragungsfunktion 
        des Systems.
    \item Welches Ausgangssignal liefert das System für $x[n] = (-1)^n\;\;\forall n$?
    \item Nun wird das System mit dem Signal $x[n] = \lambda^n$ angeregt.
        Wie ist $\lambda$ zu wählen, damit $y[n] \equiv 0$ ist?
    \item Geben Sie eine Realisierung des Systems an.
\end{enumerate}
\end{Exercise}
\begin{Answer}
    Als erstes werde ich die Formel für $y[n]$ etwas vereinfachen:
    \begin{eqnarray*}
        y[n]&=&\sum_{k=n-1}^{n+1}(x[k+1]-x[k]+x[k-1]\\
        y[n]&=&x[n-1+1]-\cancel{x[n-1]}+x[n-1-1]+\cancel{x[n+1]}-x[n]+\cancel{x[n-1]}\\
            &&\;\;+x[n+1+1]-\cancel{x[n+1]}+x[n+1-1]\\
        y[n]&=&x[n]+x[n-2]-\cancel{x[n]}+x[n+2]+\cancel{x[n]}\\
        y[n]&=&x[n-2]+x[n]+x[n+2]\\
    \end{eqnarray*}
    \begin{uebsp_theory}
        Ein zeitdiskretes System verarbeitet das Eingangssignal $x[n]$ durch
        Anwendung eines zeitdiskret arbeitenden Algorithmus
        $\mathcal{T}\{\cdot\}$ zu einem Ausgangssignal $y[n]$.
        \begin{center}
        \begin{tikzpicture}
            \matrix[row sep=0.5cm, column sep=3.5cm] {
                \node (input) {};&
                \node (T) [block] {$\mathcal{T}\{\cdot\}$};&
                \node (output) {};\\
            };
            \draw [draw,->] (input) node[above] {$x[n]$} -- (T);
            \draw [draw,->] (T) -- node[above] {$y[n]=\mathcal{T}\{x[n]\}$} (output);
        \end{tikzpicture}
        \end{center}
    \end{uebsp_theory}
    \begin{enumerate}[a)]
        \item Prüfen Sie (mit Beweis), ob das System die folgenden Eigenschaften besitzt:

            Generell gilt lt. Tutor: sobald die Impulsantwort berechenbar ist,
            handelt es sich um ein LTI-System, somit sind die 2 Bedingungen
            Linearität und Zeitinvarianz immer erfüllt, wenn die Impulsantwort
            berechenbar ist. (siehe \fref{item:impulsantwort})
            \begin{enumerate}[1)]
                \item Linearität,
                    \begin{definition}[Linearität(Superpositionsprinzip)]:\\
                        Bei einem linearen System ist die Antwort auf eine Summe
                        von Eingangssignalen gleich der Summe der
                        Einzelantworten:
                        \[y[n]=\mathcal{T}\left\{\sum_ia_ix_i[n]\right\}
                            =\sum_ia_i\mathcal{T}\left\{x_i[n]\right\}
                        =\sum_ia_iy_i[n]\]
                        \[x[n]=\sum_i a_ix_i[n]\;\;\Rightarrow\;\;y[n]=\sum_ia_iy_i[n]\]
                        Zusätzlich besitzen lin. Systeme noch folgende
                        Eigenschaft:
                        \[x[n]\equiv0\;\;\Rightarrow\;\;y[n]\equiv0\]
                    \end{definition}
                    Wir ersetzen einfach überall $x[n]$ durch $\sum_ia_ix_i[n]$:
                \begin{eqnarray*}
                    y[n]&=&x[n-2]+x[n]+x[n+2]\;\;\Rightarrow\;\;\fbox{setze $x[n]=\sum_ia_ix_i[n]$}\\
                    y[n]&=&\sum_ia_ix_i[n-2]+\sum_ia_ix_i[n]+\sum_ia_ix_i[n+2]\\
                    y[n]&=&\sum_ia_i\underbrace{(x_i[n-2]+x_i[n]+x_i[n+2])}_{y_i[n]}\\
                    y[n]&=&\sum_ia_i{y_i[n]}\;\;\Rightarrow\;\;\fbox{linear}
                \end{eqnarray*}
                Da $y[n]=\sum_ia_i{y_i[n]}$ das Ergebnis ist, das rauskommen
                muss, wenn es sich um ein lineares System handelt, muss es
                folglich linear sein.
                \item Kausalität,
                    \begin{definition}[Kausalität]:\\
                        Bei kausalen Systemen eilt die Systemantwort der
                        Systemanregung nicht vorraus. Die Systemantwort hängt
                        nur von den vergangenen Eingangssignalwerten ab.
                        \[h[n]=0\;\;\forall\;\;n<0\]
                        Wobei $h[n]$ die Impulsantwort darstellt.
                    \end{definition}
                    Da die Impulsantwort folgende ist: (Berechnung und Plot, siehe \fref{item:impulsantwort})
                    \[h[n]=\delta[n-2]+\delta[n]+\delta[n+2]\]
                    Ist leicht ersichtlich, dass $h[n]\neq0$ für $n<0$:
                    z.B: Einschalten zum Zeitpunkt $n$, dann müsste gelten: $h[n]=0\;\;\forall n<0$.
                    Das ist aber nicht erfüllt, siehe Plot.$\;\Rightarrow\;$\fbox{Akausal}
                \item Stabilität,
                    \begin{definition}[Stabilität]:\\
                        Ein System heißt stabil, wenn es auf ein
                        amplitudenbegrenztes Eingangssignal mit einem
                        amplitudenbegrenzten Ausgangssignal antwortet.
                        (BIBO...Bounded Input - Bounded Output)

                        Für zeitdiskrete LTI-Systeme gilt:
                        \[\sum_{k=-\infty}^\infty\left|h[k]\right|<\infty\]
                        $\Rightarrow$ Systeme mit endlich langer Impulsantwort sind immer stabil.

                        Wobei $h[n]$ die Impulsantwort darstellt.
                    \end{definition}
                    Da die Impulsantwort folgende ist: (Berechnung und Plot, siehe \fref{item:impulsantwort})
                    \[h[n]=\delta[n-2]+\delta[n]+\delta[n+2]\]
                    folgt:
                    \[\sum_{k=-\infty}^\infty |h[k]|=\sum_{k=-\infty}^\infty\underbrace{\delta[k-2]}_{=1\text{ für }k=2}+\underbrace{\delta[k]}_{=1\text{ für }k=0}+\underbrace{\delta[k+2]}_{=1\text{ für }k=-2}=3<\infty\;\Rightarrow\;\fbox{BIBO-stab.}\]
                \item Zeitinvarianz.
                    \begin{definition}[Zeitinvarianz]:\\
                        Bei zeitinvarianten Systemen ändert sich die Form des
                        Ausgangssignals nicht, wenn das Eingangssignal zu
                        verschiedenen Zeitpunkten angelegt wird. (Es tritt nur
                        eine Zeitverschiebung auf)
                        \[\mathcal{T}\{\delta[n-k]\}=h[n-k]\]
                        \[\mathcal{T}\{x[n-k]\}=y[n-k]\]
                    \end{definition}
                Im folgenden werden wir zuerst $x[n]$ durch $x[n-k]$ ersetzen, 
                dann werden wir dieses $x[n-k]$ in die Formel von $y[n]$
                anstatt dem $x[n]$ einsetzen.

                Als nächstes müssen wir in der Formel von $y[n]$ das $n$ durch
                $n-k$ ersetzen und mit dem obigen Ergebnis vergleichen.
                 \begin{eqnarray*}
                     \mathcal{T}\{x[n-k]\}&=&x[n-k-2]+x[n-k]+x[n-k+2]\\
                    y[n-k]&=&x[n-k-2]+x[n-k]+x[n-k+2]\\
                    &\Rightarrow&\mathcal{T}\{x[n-k]\}=y[n-k]\;\;\Rightarrow\;\;\fbox{zeitinvariant,
                 da beide gleich.}
                \end{eqnarray*}                   
            \end{enumerate}
        \item Berechnen und skizzieren der Impulsantwort und der
            Übertragungsfunktion:
            \label{item:impulsantwort}
            \begin{uebsp_theory}
                Wird am Eingang der Der Dirac-Impuls $\delta[n]$ angelegt,
                erhält man am Ausgang die Impulsantwort $h[n]$.
            \end{uebsp_theory}
            Um die Impulsantwort zu berechnen, ersetzen wir in der bereits
            vereinfachten Formel das Eingangssignal $x[n]$ durch den Impuls 
            $\delta[n]$:
            \[y[n]=x[n-2]+x[n]+x[n+2]\;\;\Rightarrow\;\;
                h[n]=\delta[n-2]+\delta[n]+\delta[n+2]\]
            \begin{center}
                \begin{tikzpicture}
                    \begin{axis}[%
                        big,
                        legend style={at={(1,1.12), anchor=north west}},
                        xlabel={$n$},
                        ylabel={$h[n]$}]
                        \addplot+[domain=-4:-1,ycomb,black,thick,mark=*,samples=4]
                        {0};
                        \addlegendentry[align=left]{$h[n]=\delta[n-2]+\delta[n]+\delta[n+2]$};
                        \addplot+[ycomb,black,thick,mark=*,forget plot,samples
                            at={-2,0,2}] {1};
                        \addplot+[ycomb,black,thick,mark=*,forget plot,samples
                            at={-6,...,-3,-1,1,3,4,...,6}] {0};
                    \end{axis}
                \end{tikzpicture}
            \end{center}
            Nun wenden wir uns der Übertragungsfunktion zu:
            \begin{uebsp_theory}
                Die Übertragungsfunktion $H(e^{j\theta})$ ist die
                Fouriertransformierte der Impulsantwort:
                \[H(e^{j\theta})=\sum_{k=-\infty}^\infty h[k]e^{-j\theta
                k}=\mathcal{FT}\{h[n]\}\]
            \end{uebsp_theory}
            Wir setzen somit die vorher berechnete Impulsantwort $h[n]$ in die
            Summe ein:
            \begin{eqnarray*}
                H(e^{j\theta})&=&\sum_{k=-\infty}^\infty h[k]e^{-j\theta k}\\
                H(e^{j\theta})&=&\sum_{k=-\infty}^\infty\left(
                    \underbrace{\delta[n-2]}_{=1\text{ für
                    }k=2}+\underbrace{\delta[n]}_{=1\text{ für
                    }k=0}+\underbrace{\delta[n+2]}_{=1\text{ für
                    }k=-2}\right)e^{-j\theta k}\\
                H(e^{j\theta})&=&e^{-j\theta -2} + \underbrace{e^{-j\theta
                    0}}_{=1} + e^{-j\theta
                    2} = e^{-j\theta 2} + 1 + e^{j\theta 2}
            \end{eqnarray*}
            \begin{definition}[Eulersche Formel]
                \[e^{j\varphi}=\cos(\varphi)+j\cdot\sin(\varphi)\]
            \end{definition}
            Mit der eulerschen Formel folgt:
            \begin{eqnarray*}
                H(e^{j\theta})&=&e^{-j\theta 2} + 1 + e^{j\theta 2}=
                    1+\underbrace{\cos(-\theta 2)}_{=\cos(\theta 2)}
                    +j\underbrace{\sin(-\theta 2)}_{=-\sin(\theta 2)}
                    +\cos(\theta 2)+j\sin(\theta 2)\\
                H(e^{j\theta})&=&
                    1+\cos(\theta 2)-\cancel{j\sin(\theta 2)}+\cos(\theta
                    2)+\cancel{j\sin(\theta 2)}=1+2\cos(\theta 2)
            \end{eqnarray*}
            \begin{center}
                \begin{tikzpicture}
                    \begin{axis}[%
                        big,
                        legend style={at={(1,1.12), anchor=north west}},
                        xtick={-6.2831, -3.1416, 0, 3.1416, 6.2831},
                        xticklabels={$-2\pi$,$-\pi$,$0$,$\pi$,$2\pi$},
                        xlabel={$n$},
                        ylabel={$h[n]$}]
                    \addplot[domain=-3*pi:3*pi, samples=101] {1+2*cos(deg(x))}; 
                    \legend{$1+2\cos(\theta 2)$}
                    \end{axis}
                \end{tikzpicture}
            \end{center}
        \item Welches Ausgangssignal liefert das System für $x[n] = (-1)^n\;\;\forall n$?
            Einsetzen von $(-1)^n$ für $x[n]$ in die vereinfachte Formel:
            \[y[n]=x[n-2]+x[n]+x[n+2]=(-1)^{n-2}+(-1)^n+(-1)^{n+2}\]
            Man kann $(-1)^{n}$ auch als cos darstellen: $\cos(n\pi)=(-1)^n$

            Mit dem cos kann man zeigen, dass gilt: 
            \[\cos((n+2z)\pi)=\cos(n\pi)\;\;\Rightarrow\;\;(-1)^{n+2z}=(-1)^{n}\;\;\;\;\forall n\in\mathbb{N},\;z\in\mathbb{Z}\]
            Da gilt: $(-1)^{n+2z}=(-1)^{n}\;\;\forall n\in\mathbb{N},\;\;z\in\mathbb{Z}$ folgt:
            \[y[n]=(-1)^{n-2}+(-1)^n+(-1)^{n+2}=(-1)^{n}+(-1)^n+(-1)^{n}=3\cdot (-1)^n\]
 
     \item Nun wird das System mit dem Signal $x[n] = \lambda^n$ angeregt.
        Wie ist $\lambda$ zu wählen, damit $y[n] \equiv 0$ ist?
             Einsetzen von $\lambda^n$ für $x[n]$ in die vereinfachte Formel:
            \begin{eqnarray*}
                y[n]&=&x[n-2]+x[n]+x[n+2]=\lambda^{n-2}+\lambda^n+\lambda^{n+2}=0\\
                0&=&\lambda^n(\lambda^{-2}+\lambda^0+\lambda^{2})\;\;\Rightarrow\;\;0=\lambda^{-2}+\lambda^0+\lambda^{2}\;\;\fbox{Multiplizieren mit: $\lambda^2$}\\
                0&=&\lambda^{0}+\lambda^2+\lambda^{4}\;\;\fbox{Subst.: $\lambda^{2x}=\alpha^{x}$}\\
                0&=&\alpha^{0}+\alpha^1+\alpha^{2}=1+\alpha+\alpha^2
            \end{eqnarray*}
            \begin{definition}[Quadratische Lösungsformel]
                \[x_{1,2}=\frac{-b\pm\sqrt{b^2-4ac}}{2a}\]
            \end{definition}
            Eingesetzt in Quadratische Lösungsformel:
            \begin{eqnarray*}
                \alpha_{1,2}&=&\frac{-1\pm\sqrt{1^2-4\cdot1\cdot1}}{2\cdot1}=\frac{-1\pm\sqrt{1-4}}{2}=\frac{-1\pm\sqrt{-3}}{2}=\frac{-1}{2}\pm\sqrt{\frac{-3}{4}}\\
                \alpha_{1,2}&=&\frac{-1}{2}\pm j\frac{\sqrt3}{2}\;\;\fbox{Rücksubst: $\alpha^{x}=\lambda^{\frac{x}{2}}=\sqrt{\lambda^x}$}\\
                \lambda_{1,2,3,4}&=&\sqrt{\frac{-1}{2}\pm j\frac{\sqrt3}{2}}\\
            \end{eqnarray*}
        \item Geben Sie eine Realisierung des Systems an.
            Hier haben wir ein Problem: ein reales System kann nur kausal sein, wir haben hier aber ein akausales System: $\Rightarrow$ $h[n]$ muss um 2 Zeiteinheiten verzögert werden und wir erhalten als neue Gleichung:
            \[y[n]=x[n]+x[n-2]+x[n-4]\]
         \begin{center}
        \begin{tikzpicture}
            \matrix[row sep=0.5cm, column sep=0.5cm] {
                &\node (splita) [point]{};&
                \node (2Ta) [block] {$2T$};&
                \node (splitb) [point]{};&
                \node (2Tb) [block] {$2T$};&
                \node (splitc) [point]{};&\\
                \node (input) {};&
                \node (splitd) [point]{};&&
                \node (suma) [sum] {$+$};&&
                \node (sumb) [sum] {$+$};&
                \node (output) {};\\
            };
            \draw [draw,->] (input)node[above] {$x[n]$} -- (splitd);
            \draw [draw,->] (splita) -- (2Ta);
            \draw [draw,->] (2Ta) -- (splitb);
            \draw [draw,->] (splitb) -- (2Tb);
            \draw [draw,->] (2Tb) -- (splitc);
            \draw [draw,->] (splitd) -- (splita);
            \draw [draw,->] (splitd) -- (suma);
            \draw [draw,->] (suma) -- (sumb);
            \draw [draw,->] (splitb) -- (suma);
            \draw [draw,->] (splitc) -- (sumb);
            \draw [draw,->] (sumb) --node[above] {$y[n]$} (output);
        \end{tikzpicture}
        \end{center}
   
    \end{enumerate}
\end{Answer}
\end{uebsp}
