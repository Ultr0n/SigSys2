\begin{uebsp}
\begin{Exercise}
Gegeben ist ein LTI–System mit der Impulsantwort
\[h[n]=12\left(\left(-\frac{1}{3}\right)^n-\left(-\frac{1}{2}\right)^n\right)\sigma[n].\]
\begin{enumerate}[a)]
    \item Berechnen und skizzieren Sie die Sprungantwort $a[n]$ des Systems.
    \item Zur Vermeidung von Überschwingen der Sprungantwort $a[n]$ wird nun zum gegebenen System ein System in Kette geschaltet. 
        Dieses Entzerrersystem wird durch die folgende Eingangs/Ausgangsbeziehung beschrieben:
        \[y[n]=ax[n]+bx[n-1]+cx[n-2].\]
        \begin{center}
            \begin{tikzpicture}
                \matrix[row sep=0.5cm, column sep=0.7cm] {
                    \node (input) {};&
                    \node (h) [block,minimum height=1.4cm] {$h[n]$};&
                    \node [above] {$x[n]$};&
                    \node (e) [block,minimum height=1.4cm]
                    {\begin{minipage}{1.5cm}\begin{center}Entzerrer\\$a,b,c$\end{center}\end{minipage}};&
                    \node (output) {};\\
                    &\node (hdesc) {gegeben}; &
                    &\node (edesc) {gesucht}; &\\
                };
                \draw [draw,->] (input) -- (h);
                \draw [draw,->] (h) -- (e);
                \draw [draw,->] (e) -- node[above] {$y[n]$} (output);
            \end{tikzpicture}
            \end{center}
        Wie sind die Koeffizienten $a$, $b$, $c$ zu wählen, so dass das
        Gesamtsystem kein Überschwingen zeigt, d.h. die Sprungantwort des
        entzerrten Gesamtsystems ist $\tilde a[n]=\sigma[n-n_0]$?

    \item Wie groß sollte zweckmäßigerweise die Zeitverzögerung $n_0$ gewählt
    werden?
\end{enumerate}
\end{Exercise}
\begin{Answer}
\begin{enumerate}[a)]
    \item Berechnen und skizzieren Sie die Sprungantwort $a[n]$ des Systems:
    \begin{uebsp_theory}
        Das Ausgangssignal $y[n]$ eines linearen, zeitinvarianten Systems mit der Sprungantwort $h[n]$ und dem Eingangssignal $x[n]$ kann mit folgender Formel berechnet werden:
        \begin{equation}
            y[n]=\sum_{k=-\infty}^\infty x[k]h[n-k]=\sum_{k=-\infty}^\infty
            x[n-k]h[k]
            \label{eq:ausgang_eingang}
        \end{equation}
    \end{uebsp_theory}
    \begin{uebsp_theory}
        Ist das Eingangssignal eines Systems mit der Sprungantwort $h[n]$ die 
        Sprungfunktion $\sigma[n]$, dann erhält man am Ausgang die Sprungantwort:

        (Eingesetzt für $x[n]=\sigma[n]$ in \fref{eq:ausgang_eingang} erhält man somit)
        \[a[n]=\sum_{k=-\infty}^\infty \sigma[k]h[n-k]=\sum_{k=-\infty}^\infty
        \sigma[n-k]h[k]\]
    \end{uebsp_theory}
    \begin{definition}[Geometrische Reihe]
        \[\sum_{n=0}^\infty a^n=\frac{1}{1-a}\;\;\forall
            |a|<1\;\;\;\;\text{ und }\;\;\;\;
            \sum_{n=0}^{N-1} a^n=\begin{cases}N&a=1\\
                \frac{1-a^N}{1-a}&\text{sonst }\forall N\geq0\end{cases}\]
    \end{definition}
    \begin{eqnarray*}
        a[n]&=&\sum_{k=-\infty}^\infty\sigma[n-k]h[k]=
            \sum_{k=-\infty}^\infty\sigma[n-k]12\left(\left(-\frac{1}{3}\right)^k-\left(-\frac{1}{2}\right)^k\right)\sigma[k]\\
        a[n]&=&12\left(\sum_{k=-\infty}^\infty\underbrace{\sigma[k]}_{=1\text{
            für }k\geq0\;}\underbrace{\sigma[n-k]}_{\;=1\text{ für }n\geq
            k}\left(\left(-\frac{1}{3}\right)^k-\left(-\frac{1}{2}\right)^k\right)\right)
            \fbox{\begin{minipage}{0.2\linewidth}Somit
                geht $\sum$ von $k=0$ bis $n$\end{minipage}}\\
        a[n]&=&12\left(\underbrace{\sum_{k=0}^n\left(-\frac{1}{3}\right)^k}_{Geom. Reihe}
            -\underbrace{\sum_{k=0}^n\left(-\frac{1}{2}\right)^k}_{Geom. Reihe}\right)\\
        a[n]&=&12\left(\frac{1-\left(-\frac{1}{3}\right)^{n+1}}{1+\frac{1}{3}}
            -\frac{1-\left(-\frac{1}{2}\right)^{n+1}}{1+\frac{1}{2}}\right)\;\;\;\forall
            n\geq0\;\Rightarrow\fbox{Mit $\sigma[n]$ multiplizieren}\\
        a[n]&=&12\left(\frac{1-\left(-\frac{1}{3}\right)^{n+1}}{\frac{4}{3}}
            -\frac{1-\left(-\frac{1}{2}\right)^{n+1}}{\frac{3}{2}}\right)\sigma[n]\\
        a[n]&=&12\left(\frac{3}{4}\left(1-\left(-\frac{1}{3}\right)^{n+1}\right)
            -\frac{2}{3}\left(1-\left(-\frac{1}{2}\right)^{n+1}\right)\right)\sigma[n]\\
        a[n]&=&\cancel{12}\left(\frac{9\left(1-\left(-\frac{1}{3}\right)^{n+1}\right)
            -8\left(1-\left(-\frac{1}{2}\right)^{n+1}\right)}{\cancel{12}}\right)\sigma[n]\\
        a[n]&=&\left(1-9\left(-\frac{1}{3}\right)^{n+1}
            +8\left(-\frac{1}{2}\right)^{n+1}\right)\sigma[n]\\
        a[n]&=&\left(1-\cancel9\left(-\frac{1}{3}\right)^{n-1}\frac{1}{\cancel9}
            +\cancel8\left(-\frac{1}{2}\right)^{n-2}\left(-\frac{1}{\cancel8}\right)\right)\sigma[n]\\
        a[n]&=&\left(1-\left(-\frac{1}{3}\right)^{n-1}
            -\left(-\frac{1}{2}\right)^{n-2}\right)\sigma[n]
    \end{eqnarray*}
        \def\lbl{$a[n]=(1-\left(-\frac{1}{3}\right)^{n-1}
            -\left(-\frac{1}{2}\right)^{n-2})\sigma[n]$}
        \begin{center}
            \begin{tikzpicture}
                \begin{axis}[%
                    big,
                    legend style={at={(1,1.12), anchor=north west}},
                    xlabel={$n$},
                    ylabel={$y[n]$}]
                    \addplot+[domain=-4:-1,ycomb,black,thick,mark=*,samples=4]
                    {0};
                    \addlegendentry[align=left]{\lbl};
                    \addplot+[domain=0:15,ycomb,black,thick,mark=*,forget
                        plot,samples=16] {1-(-1/3)^(x-1)-(-1/2)^(x-2)};
                \end{axis}
            \end{tikzpicture}
        \end{center}
        \item Berechnen der Koeffizienten $a$, $b$, $c$, um Überschwingen zu
        vermeiden ($\tilde a[n]=\sigma[n-n_0]$)
        {\footnotesize
        \begin{eqnarray*}
            y[n]&=&\tilde a[n]=\sigma[n-n_0]=ax[n]+bx[n-1]+cx[n-2]\\
            \sigma[n-n_0]&=&a\left(1-\left(-\frac{1}{3}\right)^{n-1}-\left(-\frac{1}{2}\right)^{n-2}\right)\sigma[n]
                +b\left(1-\left(-\frac{1}{3}\right)^{n-2}-\left(-\frac{1}{2}\right)^{n-3}\right)\sigma[n-1]\\
                &&\;\;+c\left(1-\left(-\frac{1}{3}\right)^{n-3}-\left(-\frac{1}{2}\right)^{n-4}\right)\sigma[n-2]
        \end{eqnarray*}}
        Das Promblem sind nun die lästigen $\sigma$-Sprünge, die Zu den
        Zeitpunkten $n=0,1,2$ und $n_0$ auftreten. 

        Wenn wir jedoch ein sehr großes $n$ betrachten ($n>n_0$ und $n>2$), dann
        muss logischerweise die Gleichung noch genauso gelten, jedoch haben wir
        den Vorteil, dass wir uns das $\sigma$ wegdenken können, denn es
        besitzen bereits alle $\sigma$-Sprünge den Wert $1$.

        Somit gilt $\forall n>n_0$ und $\forall n>2$ folgendes:
        \begin{eqnarray*}
            1&=&a\left(1-\left(-\frac{1}{3}\right)^{n-1}-\left(-\frac{1}{2}\right)^{n-2}\right)\cdot 1
                +b\left(1-\left(-\frac{1}{3}\right)^{n-2}-\left(-\frac{1}{2}\right)^{n-3}\right)\cdot 1\\
                &&\;\;+c\left(1-\left(-\frac{1}{3}\right)^{n-3}-\left(-\frac{1}{2}\right)^{n-4}\right)\cdot
                1\\
            1&=&a+b+c-a\left(-\frac{1}{3}\right)^{n-3}\frac{1}{3^2}-b\left(-\frac{1}{3}\right)^{n-3}\left(-\frac{1}{3}\right)-c\left(-\frac{1}{3}\right)^{n-3}\\
            &&\;\;-a\left(-\frac{1}{2}\right)^{n-4}\frac{1}{2^2}-b\left(-\frac{1}{2}\right)^{n-4}\left(-\frac{1}{2}\right)-c\left(-\frac{1}{2}\right)^{n-4}\\
            1&=&a+b+c-\left(\frac{a}{9}-\frac{b}{3}+c\right)\left(-\frac{1}{3}\right)^{n-3}
                -\left(\frac{a}{4}-\frac{b}{2}+c\right)\left(-\frac{1}{2}\right)^{n-4}
        \end{eqnarray*}
        Nun können wir mittels Koeffizientenvergleich die Werte für $a$, $b$ und
        $c$ bestimmen (das geht allerdings nur, weil
        $\left(-\frac{1}{3}\right)^{n-3}$ und $\left(-\frac{1}{2}\right)^{n-4}$
        linear unabhängig sind)
        \begin{enumerate}[i)]
            \item $\begin{aligned}[t]1=a+b+c\end{aligned}$
            \item $\begin{aligned}[t]0=\left(\frac{a}{9}-\frac{b}{3}+c\right)\left(-\frac{1}{3}\right)^{n-3}\;\;\Rightarrow\;\;
                0=\left(\frac{a}{9}-\frac{b}{3}+c\right)\;\;\Rightarrow\;\;
                b=3\left(\frac{a}{9}+c\right)\end{aligned}$
            \item
            $\begin{aligned}[t]0=\left(\frac{a}{4}-\frac{b}{2}+c\right)\left(-\frac{1}{2}\right)^{n-4}\;\;\Rightarrow\;\;
                0=\left(\frac{a}{4}-\frac{b}{2}+c\right)\;\;\Rightarrow\;\;
                b=2\left(\frac{a}{4}+c\right)\end{aligned}$
        \end{enumerate}
        Somit erhalten wir:
        \[
            2\left(\frac{a}{4}+c\right)=3\left(\frac{a}{9}+c\right)\;\;\Rightarrow\;\;
            \frac{a}{2}+2c=\frac{a}{3}+3c\;\Rightarrow\;
            \frac{3a}{6}-\frac{2a}{3}=c\;\Rightarrow\;
            \frac{a}{6}=c\;\Rightarrow\;a=6c
        \]
        Somit können wir $c$ berechnen:
        \[1=a+\underbrace{b}_{\frac{a}{2}+2c}+c=a+\underbrace{\frac{a}{2}}_{3c}+2c+c=6c+3c+3c=12c\;\;\Rightarrow\;\;c=\frac{1}{12}\]
        Somit können wir $a$ berechnen:
        \[a=6c=\frac{6}{12}=\frac{1}{2}\]
        Somit können wir $b$ berechnen:
        \[b=\frac{a}{2}+2c=\frac{1}{4}+\frac{2}{12}=\frac{5}{12}\]

        Somit erhalten wir als neue Gleichung:
        \begin{eqnarray}
            \sigma[n-n_0]&=&\frac{1}{2}\left(1-\left(-\frac{1}{3}\right)^{n-1}-\left(-\frac{1}{2}\right)^{n-2}\right)\cdot
            \sigma[n]\\
                &&\;\;+\frac{5}{12}\left(1-\left(-\frac{1}{3}\right)^{n-2}-\left(-\frac{1}{2}\right)^{n-3}\right)\cdot
                \sigma[n-1]\\
                &&\;\;+\frac{1}{12}\left(1-\left(-\frac{1}{3}\right)^{n-3}-\left(-\frac{1}{2}\right)^{n-4}\right)\cdot
                \sigma[n-2]
            \label{eq:komponenten}
        \end{eqnarray}

        \item Gesucht ist nun die Zeitverzögerung $n_0$:
            Durch $\sigma[n], \sigma[n-1]$ und $\sigma[n-2]$ folgt, dass
            $n_0\geq 0$ sein muss.
            Wenn wir uns nun die Komponenten in \fref{eq:komponenten} ansehen:\\
            $\begin{aligned}[t]A=\frac{1}{2}\left(1-\left(-\frac{1}{3}\right)^{n-1}-\left(-\frac{1}{2}\right)^{n-2}\right)\cdot
                \sigma[n]\end{aligned}$\\
            $\begin{aligned}[t]B=\frac{5}{12}\left(1-\left(-\frac{1}{3}\right)^{n-2}-\left(-\frac{1}{2}\right)^{n-3}\right)\cdot
                \sigma[n-1]\end{aligned}$\\
            $\begin{aligned}[t]C=\frac{1}{12}\left(1-\left(-\frac{1}{3}\right)^{n-3}-\left(-\frac{1}{2}\right)^{n-4}\right)\cdot
                \sigma[n-2]\end{aligned}$\\
           Dann können wir uns die ersten paar Werte berechnen:
           \begin{center}
           \begin{tabular}{c|cccccc}
            n&  $0$&  $1$    &$2$    &$3$  &$4$ &$\cdots$\\
            \hline
            A&$0$&$1$&$\frac{1}{6}$&$\frac{25}{36}$&$\frac{85}{216}$&$\cdots$\\
            B&$0$&$0$&$\frac{5}{6}$&$\frac{5}{36}$&$\frac{125}{216}$&$\cdots$\\
            C&$0$&$0$&$0$&$\frac{6}{36}$&$\frac{6}{216}$&$\cdots$\\
            \hline
            $\sum$&$0$&$1$&$1$&$1$&$1$&$\cdots$\\
           \end{tabular}
           \end{center}
           Somit sieht man gut, dass $\sum=(A+B+C)$ ab $n\geq1$ immer $1$ ergibt.
           
           Folglich muss $n_0=1$ sein.

        \begin{center}
            \begin{tikzpicture}
                \def\lbl{$\sigma[n-n_0]=\sigma[n-1]$}

                \begin{axis}[%
                    big,
                    legend style={at={(1,1.12), anchor=north west}},
                    xlabel={$n$},
                    ylabel={$y[n]$}]
                    \addplot+[domain=-4:0,ycomb,black,thick,mark=*,samples=5]
                    {0};
                    \addlegendentry[align=left]{\lbl};
                    \addplot+[domain=1:10,ycomb,black,thick,mark=*,samples=10]
                    {1};
                \end{axis}
            \end{tikzpicture}
        \end{center}

        \textbf{TODO: Wie kommt man darauf, dass der Entzerrer aus 3 Komponenten
        besteht? Kann er auch aus 2 oder 1ner Komponente bestehen?}
    \end{enumerate}
\end{Answer}
\end{uebsp}
