\begin{uebsp}
\begin{Exercise}
Das Signal $x[n]$ sei periodisch mit der Periode $N$. Prüfen Sie, ob das Signal $y[n] = x[M n]$
($M$ ganzzahlig) ebenfalls periodisch ist. Bestimmen Sie gegebenenfalls die Periode von $y[n]$.
\end{Exercise}
\begin{Answer}
    \begin{uebsp_theory}
    Ein Signal heißt \textbf{Periodisch}, mit der Periodendauer $N\in\mathbb{Z}$, wenn gilt: (bei beliebigem $m\in \mathbb{Z}$)
    \[x[n]=x[n+mN]\]
    \end{uebsp_theory}

    Angenommen, wir definieren 2 Variablen:\\
    $n\in\mathbb{N}$ und $o=2n$, dann ist klar, dass $o\in\mathbb{Z}$ ebenfalls gilt.\\
    
    Nehmen wir nun allgemein die Variable $M\in\mathbb{Z}$ und definieren $l=Mn$, dann gilt auch $l\in\mathbb{Z}$.\\

    Somit gilt folgendes: (denn die gesamten Variablen $n$, $o$ und $l$ liegen ja im Zahlenbereich von $\mathbb{Z}$.)
    \[x[n]=x[n+m_xN_x]\;\;\text{und}\;\;x[o]=x[o+m_xN_x]\;\;\text{und}\;\;x[l]=x[l+m_xN_x]\]
    Das bedeutet, dass $x[n]$, $x[o]$ und $x[l]$ periodische Signale sind.
    
    Somit muss $y[n]$ auch ein periodisches Signal sein, denn es gilt:
    \[x[l]=x[Mn]=y[n]=y[n+m_yN_y]\]

    Zum Bestimmen der Periode überlegen wir uns folgendes:
    \[x[Mn+m_xN_x]=y[n+m_yN_y]\]

\end{Answer}
\end{uebsp}
