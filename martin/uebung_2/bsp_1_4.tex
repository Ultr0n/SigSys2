\begin{uebsp}
\begin{Exercise}
    Welche Grundperiode hat das zeitdiskrete Signal
    \[x[n]=e^{j\frac{2\pi}{N}nk}\]
    in Abhängigkeit von den ganzzahligen Größen $k$ und $N$?
\end{Exercise}
\begin{Answer}
\begin{uebsp_theory}
    Ein Signal heißt \textbf{Periodisch}, mit der Periodendauer $N\in\mathbb{Z}$, wenn gilt: (bei beliebigem $m\in \mathbb{N}$)
    \[x[n]=x[n+mN]\]
\end{uebsp_theory}
\begin{uebsp_theory}
    Die Grundperiode ist definiert, als die kleinste Periode unter der Menge aller möglichen Perioden von $x[n]$.
\end{uebsp_theory}

Ich definiere am Anfang ein $m = 1$.

Weiters definiere ich, dass $N = m\cdot Nx$ und $k = m\cdot kx$. Wobei in beiden Gleichungen ebenfalls ersichtlich ist, dass $N_x, k_x \in \mathbb{Z}$. (denn laut Angabe ist auch $k\in\mathbb{N}$ und $N\in\mathbb{N}$.
Somit lautet der Exponent:
\[x[n]=e^{j\frac{2\pi}{N}nk}=e^{j\frac{2\pi}{N_x\cdot m}nk_x\cdot m}\]

Dass dies gilt, ist ersichtlich, denn $m=1$ ist ein gemeinsamer Teiler aller Zahlen $\mathbb N$.\\

Ohne Beschränkung der Allgemeinheit können wir nun einen Schritt weiter gehen und $m$ als den gemeinsamen Teiler von $N$ und $k$ definieren. Somit können wir sagen:
\[e^{j\frac{2\pi}{N}nk=e^{j\frac{2\pi}{N_x\cdot m}nk_x\cdot m}}\]
Somit bekommen wir eine neue $e$-Funktion:
\[e^{j\frac{2\pi}{N}nk=e^{j\frac{2\pi}{N_x\cdot \text{ggT}(N,k)}nk_x\cdot \text{ggT}(N,k)}}\]
Da gilt:
\[N=m\cdot N_x=\text{ggT}(N,k)N_x\]
Folgt:
\[\underline{\underline{N_x=\frac{N}{\text{ggT}(N,k)}}}\]

Für den Zusammenhang zwischen ggT und kgV gilt in diesem Fall:
\[ggT\cdot kgV=k\cdot N\;\;\Rightarrow\;\;\underline{\underline{N_x=\frac{kgV(N,k)}{k}}}\]
\begin{bsp}
$k=70$, $N=120$, die Primfaktorzerlegung ergibt folgendes:\\
\begin{center}
\begin{tabular}{c|c|c}
&$k=70$&$N=120$\\
\hline
2&{\color{red}1}&{\color{green}3}\\
3&{\color{red}0}&{\color{green}1}\\
5&{\color{red}1}&{\color{green}1}\\
7&{\color{green}1}&{\color{red}0}\\
\hline
\end{tabular}
\end{center}
D.h. $k=2\cdot5\cdot7=70$ und $N=2^3\cdot3\cdot5=120$
Somit berechnet sich der {\color{red}ggT} als das kleinere der beiden Zahlen in der Tabelle:
\[\text{ggT}(N,k)=2^1\cdot 3^0\cdot 5^1\cdot 7^0=10\]

Und das {\color{green}kgV} als das größere der beiden Zahlen in der Tabelle:
\[\text{kgV}(N,k)=2^3\cdot 3^1\cdot 5^1\cdot 7^1=840\]
\end{bsp}

\begin{hint}
Der Weg vom Tutor war der folgende:
\begin{eqnarray*}
e^{j\frac{2\pi}{N}nk}&=&e^{j\frac{2\pi}{N}(n+N_x)k}\\
\cancel{e^{j\frac{2\pi}{N}nk}}&=&\cancel{e^{j\frac{2\pi}{N}nk}}\cdot e^{j\frac{2\pi}{N}N_xk}\\
1&=&e^{j\frac{2\pi}{N}N_xk}
\end{eqnarray*}
Wobei hier ersichtlich sein sollte, dass $\frac{1}{N}N_xk\in\mathbb{N}$ ist.
\[N_x\cdot\frac{\frac{k}{\text{ggT}(k,N)}}{\frac{N}{\text{ggT}(k,N)}}\]

\textbf{TODO: keine ahnung was das bringt, hab aber auch keinen bock mehr!!!}
\end{hint}
\end{Answer}
\end{uebsp}
