\begin{uebsp}
\begin{Exercise}
    Welche Grundperiode hat das zeitdiskrete Signal
    \[x[n]=e^{j\frac{2\pi}{N}nk}\]
    in Abhängigkeit von den ganzzahligen Größen $k$ und $N$?
\end{Exercise}
\begin{Answer}
Ich definiere am Anfang ein $m = 1$.

Weiters definiere ich, dass $N = m\cdot Nx$ und $k = m\cdot kx$. Wobei in beiden Gleichungen ebenfalls ersichtlich ist, dass $N_x, k_x \in \mathbb{N}$.
Somit lautet der Exponent:
\[e^{j\frac{2\pi}{N}nk}=e^{j\frac{2\pi}{N_x\cdot m}nk_x\cdot m}\]

Dass dies gilt, ist ersichtlich, denn $m=1$ ist ein gemeinsamer Teiler aller Zahlen $\mathbb N$.\\

Wenn wir nun einen Schritt weiter gehen und $m$ definieren, als den gemeinsamen Teiler von $N$ und $k$, können wir genauso sagen:
\[e^{j\frac{2\pi}{N}nk=e^{j\frac{2\pi}{N_x\cdot m}nk_x\cdot m}}\]
Somit bekommen wir eine neue $e$-Funktion:
\[e^{j\frac{2\pi}{N}nk=e^{j\frac{2\pi}{N_x\cdot \text{ggT}(N,k)}nk_x\cdot \text{ggT}(N,k)}}\]
Da gilt:
\[N=m\cdot N_x=\text{ggT}(N,k)N_x\]
Folgt:
\[N_x=\frac{N}{\text{ggT}(N,k)}\]
\end{Answer}
\end{uebsp}
