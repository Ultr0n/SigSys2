\begin{uebsp}
\begin{Exercise}
    Die Fourierreihendarstellungen der periodischen Signale $x[n]$ und $y[n]$ sind gegeben durch
    \[x[n]=\sum_{k=0}^{N-1}c_ke^{j\frac{2\pi}{N}nk}\Leftrightarrow c_k=\frac{1}{N}\sum_{n=0}^{N-1}x[n]e^{-j\frac{2\pi}{N}nk}\]
    und 
    \[y[n]=\sum_{k=0}^{N-1}d_ke^{j\frac{2\pi}{N}nk}\Leftrightarrow d_k=\frac{1}{N}\sum_{n=0}^{N-1}y[n]e^{-j\frac{2\pi}{N}nk}\]
    Berechnen Sie den Zusammenhang zwischen den Fourierreihenkoeffizienten $d_k$ und $c_k$ für die folgenden Signalbeziehungen:
    \begin{enumerate}[a)]
        \item $y[n]=x\left[n-\frac{N}{2}\right],\;\;N$ gerade
        \item $y[n]=x\left[N-n\right]$
        \item $y[n]=\frac{1}{2}\left(x[n]+x^*\left[N-n\right]\right)$
        \item $\cancel{y[n]=x[2n]}$
        \item $y[n]=x[n]\cos\frac{2\pi L}{M}n,\;\;L$ und $M$ ganzzahlig
        \item $y[n]=x^2[n]$
    \end{enumerate}
\end{Exercise}
\begin{Answer}
     \begin{enumerate}[a)]
        \item $y[n]=x\left[n-\frac{N}{2}\right],\;\;N$ gerade
            \begin{definition}[Eulersche Formel]
                \[e^{j\varphi}=\cos(\varphi)+j\sin(\varphi)\]
            \end{definition}
            \begin{uebsp_theory}
                Mit der Eulerschen Formel können wir uns folgedes herleiten:
                \[e^{j\cdot\pi}=\underbrace{\cos(\pi)}_{=-1}+\underbrace{j\sin(\pi)}_{=0}=-1\]
            \end{uebsp_theory}

        \begin{eqnarray*}
            y[n]&=&x\left[n-\frac{N}{2}\right]=\sum_{k=0}^{N-1}c_k\cdot e^{j\frac{2\pi}{N}\left(n-\frac{N}{2}\right)k}=
            \sum_{k=0}^{N-1}c_k\cdot e^{j\frac{2\pi}{N}nk}\cdot e^{-j\frac{\cancel 2\pi}{\cancel N}\left(\frac{\cancel N}{\cancel 2}\right)k}\\
        y[n]&=&\sum_{k=0}^{N-1}c_k\cdot e^{j\frac{2\pi}{N}nk}\cdot (\underbrace{e^{-j\pi}}_{-1})^k=
            \underbrace{\sum_{k=0}^{N-1}c_k\cdot e^{j\frac{2\pi}{N}nk}}_{x[n]}\cdot -1^k=(-1)^k\cdot x[n]\\
            \sum_{k=0}^{N-1}d_k\cancel{e^{j\frac{2\pi}{N}nk}}&=&(-1)^k\sum_{k=0}^{N-1}c_{k}\cdot \cancel{e^{j\frac{2\pi}{N}k\cdot n}}\;\;\Rightarrow\;\;
            \underbrace{\sum_{k=0}^{N-1}d_{k}}_{N\cdot d_{k}}=(-1)^k\underbrace{\sum_{k=0}^{N-1}c_{k}}_{N\cdot c_{k}}\;\;\\
            \cancel N\cdot d_k&=&(-1)^k\cancel N\cdot c_{k}\;\;\Rightarrow\;\;
            \underline{\underline{d_k=c_{k}(-1)^k}}
        \end{eqnarray*}
        \begin{hint}
            Hier gibts in der Formelsammlung(Fourierreihen zeitdiskreter periodischer Signale) eine nette Formel, die besagt:
            \[x[n-N_0]\;\;\Leftrightarrow\;\;e^{-j\frac{2\pi k}{N}N_0}\cdot c_k\]
            Somit kann man das ganze etwas abkürzen, indem man sagt:
            \[x[n-\frac{N}{2}]\;\;\Leftrightarrow\;\;e^{-j\frac{\cancel 2\pi k}{\cancel N}\frac{\cancel N}{\cancel 2}}\cdot c_k=\underbrace{e^{-j\pi k}}_{=(-1)^k}\cdot c_k=c_k\cdot (-1)^k\]
        \end{hint}
        \item $y[n]=x\left[N-n\right]$
            \begin{uebsp_theory}
                Mit der Eulerschen Formel können wir uns folgedes herleiten:
                \[e^{j\cdot 2\cdot\pi}=\underbrace{\cos(2\cdot\pi)}_{=1}+\underbrace{j\sin(2\cdot\pi)}_{=0}=1\]
            \end{uebsp_theory}

        \begin{eqnarray*}
            y[n]&=&x\left[N-n\right]=\sum_{k=0}^{N-1}c_k\cdot e^{j\frac{2\pi}{N}\left(N-n\right)k}=
            \sum_{k=0}^{N-1}c_k\cdot e^{-j\frac{2\pi}{N}k\cdot n}\cdot e^{j\frac{2\pi}{\cancel N}\cancel N\cdot k}\\
            y[n]&=&\sum_{k=0}^{N-1}c_k\cdot e^{-j\frac{2\pi}{N}k\cdot n}\cdot (\underbrace{e^{j\cdot 2\pi}}_{=1})^k=\sum_{k=0}^{N-1}c_k\cdot e^{-j\frac{2\pi}{N}k\cdot n}\;\;\fbox{subst: $k=-k$}\\
            y[n]&=&\sum_{k=0}^{N-1}c_{-k}\cdot e^{j\frac{2\pi}{N}k\cdot n}\\
            \sum_{k=0}^{N-1}d_k\cancel{e^{j\frac{2\pi}{N}nk}}&=&\sum_{k=0}^{N-1}c_{-k}\cdot \cancel{e^{j\frac{2\pi}{N}k\cdot n}}\;\;\Rightarrow\;\;
            \underbrace{\sum_{k=0}^{N-1}d_{k}}_{N\cdot d_{k}}=\underbrace{\sum_{k=0}^{N-1}c_{-k}}_{N\cdot c_{-k}}\;\;\Rightarrow\;\;
            \cancel N\cdot d_k=\cancel N\cdot c_{-k}\\
            &\Rightarrow& \underline{\underline{d_k=c_{-k}}}
        \end{eqnarray*}
        \begin{hint}In Formelsammlung(Fourierreihen zeitdiskreter periodischer Signale) steht:
                \[x[-n]\Leftrightarrow c_{-k}\]
                Somit hätte man sich die ganze Berechnung erspart, wenn man erkannt hätte, dass 
                \[x[-n]=x[N-n]\;\;\Leftrightarrow\;\;c_{-k}\]
                denn das Signal ist ja sowieso $N$-Periodisch.
            \end{hint}

        \item $y[n]=\frac{1}{2}\left(x[n]+x^*\left[N-n\right]\right)$
            \begin{uebsp_theory}
                $x^*[n]$ ist ein konjugiert komplexes Signal:
                \[c_{-k}^*=c_k\]
            \end{uebsp_theory}
            Da gilt: $x[-n]=x[-n+mN]\;\;\forall m\in\mathbb{Z}$ folgt:
            \begin{eqnarray*}
                y[n]=\frac{1}{2}(x[n]+x^*[N-n])\;\Rightarrow\;y[n]=\frac{1}{2}(x[n]+x^*[-n])
            \end{eqnarray*}
            \begin{uebsp_theory}
                Lt. Formel in Formelsammlung(Fourierreihen zeitdiskreter periodischer Signale) gilt:
                \[x_e[n]=\frac{1}{2}\left(x[n]+x^*[-n]\;\right)\Leftrightarrow\;\Re e\{c_k\}\]
            \end{uebsp_theory}
            Somit gilt:
            \begin{eqnarray*}
                \sum_{k=0}^{N-1}d_k\cancel{e^{j\frac{2\pi}{N}nk}}&=&\sum_{k=0}^{N-1}\Re e\{c_k\}\cdot \cancel{e^{j\frac{2\pi}{N}k\cdot n}}\;\;\Rightarrow\;\;
                \underbrace{\sum_{k=0}^{N-1}d_{k}}_{N\cdot d_{k}}=\underbrace{\sum_{k=0}^{N-1}\Re e\{c_k\}}_{N\cdot \Re e\{c_k\}}\\
                \Rightarrow\cancel N\cdot d_k&=&\cancel N\cdot \Re e\{c_k\}\;\;\Rightarrow\;\;
                \underline{\underline{d_k=\Re e\{c_{k}\}}}
            \end{eqnarray*}
        \begin{hint}
            Hier gibts in der Formelsammlung(Fourierreihen zeidiskreter periodischer Signale) eine nette Formel, die besagt:
            \[x_e[n]=\frac{1}{2}(x[n]+x^*[-n])\;\;\Leftrightarrow\;\;\Re e\{c_k\}\]
            Somit kann man das ganze etwas abkürzen, indem man sagt:
            \[y[n]=\frac{1}{2}(x[n]+\underbrace{x^*[N-n]}_{x^*[-n]})=\frac{1}{2}(x[n]+{x^*[-n]})\;\;\Leftrightarrow\;\;\Re e\{c_k\}\]
            Da die Funktion $y[n]$ sowieso $N$-periodisch ist.
        \end{hint}

        \stepcounter{enumi}%increment counter by one
        \item $y[n]=x[n]\cos\frac{2\pi L}{M}n,\;\;L$ und $M$ ganzzahlig
            \begin{uebsp_theory}
                Nicht vergessen auf folgende Beziehung: \[\cos x=\frac{1}{2}\left(e^{jx}+e^{-jx}\right)\]
            \end{uebsp_theory}
            \begin{eqnarray*}
                y[n]&=&x[n]\cdot \cos\frac{2\pi L}{M}n=
                x[n]\cdot\frac{1}{2}\left(e^{j\frac{2\pi L}{M}n}+e^{-j\frac{2\pi L}{M}n}\right)\\
                \Rightarrow\;\;d_k&=&\frac{1}{N}\sum_{n=0}^{N-1}y[n]e^{-j\frac{2\pi}{N}nk}=
                \frac{1}{N}\sum_{n=0}^{N-1}x[n]\cdot\frac{1}{2}\left(e^{j\frac{2\pi L}{M}n}+e^{-j\frac{2\pi L}{M}n}\right)e^{-j\frac{2\pi}{N}nk}\\
                d_k&=&\frac{1}{2}\frac{1}{N}\sum_{n=0}^{N-1}x[n]\cdot\left(e^{j\frac{2\pi L}{M}n}+e^{-j\frac{2\pi L}{M}n}\right)e^{-j\frac{2\pi}{N}nk}\\
                d_k&=&\frac{1}{2}\frac{1}{N}\sum_{n=0}^{N-1}x[n]\cdot\left(e^{j2\pi n(\frac{L}{M}-\frac{k}{N})}+e^{-j2\pi n(\frac{L}{M}+\frac{k}{N})}\right)\\
            d_k&=&\frac{1}{2}\frac{1}{N}\sum_{n=0}^{N-1}x[n]\cdot e^{j2\pi n(\frac{L}{M}-\frac{k}{N})}+\sum_{n=0}^{N-1}x[n]\cdot e^{-j2\pi n(\frac{L}{M}+\frac{k}{N})}\\
                d_k&=&\frac{1}{2}\left(\underbrace{\frac{1}{N}\sum_{n=0}^{N-1}x[n]\cdot e^{-\frac{j2\pi n}{N}(k-\frac{LN}{M})}}_{c_{k-\frac{LN}{M}}}+\underbrace{\frac{1}{N}\sum_{n=0}^{N-1}x[n]\cdot e^{-\frac{j2\pi n}{N}(k+\frac{LN}{M})}}_{c_{k+\frac{LN}{M}}}\right)\\
                   &\Rightarrow & \underline{\underline{d_k=\frac{1}{2}\left({c_{k-\frac{LN}{M}}}+{c_{k+\frac{LN}{M}}}\right)}}
            \end{eqnarray*}
        \item $y[n]=x^2[n]$
            \begin{eqnarray*}
                y[n]&=&x^2[n]=x[n]\cdot x[n]= \sum_{k=0}^{N-1}\left(c_ke^{j\frac{2\pi}{N}nk}\right)\cdot \sum_{l=0}^{N-1}\left(c_le^{j\frac{2\pi}{N}nl}\right)=\\
                y[n]&=& \sum_{k=0}^{N-1}\sum_{l=0}^{N-1}\left(c_ke^{j\frac{2\pi}{N}nk}\right)\cdot \left(c_le^{j\frac{2\pi}{N}nl}\right)=\sum_{k=0}^{N-1}\sum_{l=0}^{N-1}c_k\cdot c_l\cdot e^{j\frac{2\pi}{N}n(k+l)}
            \end{eqnarray*}
            Wenn wir nun $k+l$ durch $m$ substituieren: ($k+l=m$) und diese Formel auf $k$ umstellen ($k=m-l$):
            Nicht vergessen, auch die Limits der Summen müssen angepasst werden:\\

            \textbf{Unteres Limit}: $k=m-l\;\;\Rightarrow\;\;0=m-l\;\;\Rightarrow\;\;l=m$\\
            \textbf{Oberes Limit}: $k=m-l\;\;\Rightarrow\;\;N-1=m-l\;\;\Rightarrow\;\;N-1+l=m$
            \begin{eqnarray*}
                y[n]&=& \sum_{m=l}^{N-1+l}\sum_{l=0}^{N-1}c_{m-l}\cdot c_l\cdot e^{j\frac{2\pi}{N}n\cdot m}\\
                \sum_{m=0}^{N-1}d_m\cancel{e^{j\frac{2\pi}{N}nm}}&=& \sum_{m=l}^{N-1+l}\sum_{l=0}^{N-1}c_{m-l}\cdot c_l\cdot \cancel{e^{j\frac{2\pi}{N}n\cdot m}}\;\Rightarrow\;\;
                \cancel{\sum_{m=0}^{N-1}}d_m= \cancel{\sum_{m=l}^{N-1+l}}\sum_{l=0}^{N-1}c_{m-l}\cdot c_l\\
                &\Rightarrow& \underline{\underline{d_m=\sum_{l=0}^{N-1}c_{m-l}\cdot c_l}}
            \end{eqnarray*}
    \end{enumerate}
\end{Answer}
\end{uebsp}
